\Ticket{1}

\TaskTag[todo]{№1.1}
\begin{cond}
	Решение с помощью формулы Даламбера краевых задач для полуограниченной струны.
\end{cond}

\TaskTag[todo]{№1.2}
\begin{cond}
	Пространство обобщённых функций $D'(\Omega)$. Примеры: регулярные обобщённые функции, $\delta$-функция.
\end{cond}

\TaskTag[todo]{№1.3}
\begin{cond}
	Решить краевую задачу для ограниченной струны:

	\(
	\begin{cases}
		u_{tt} = 9u_{xx} + (2t - 1)\sin \pi3x,           \\
		u|_{x=0} = u|_{x=1} = 0, & t > 0,\; x \in (0,1), \\
		u|_{t=0} = -\sin 2\pi x,\quad u_t|_{t=0} = 0.
	\end{cases}
	\)
\end{cond}

\TaskTag[progress]{№1.4}
\begin{cond}
	Найти решение задачи Коши:

	\(
	\begin{cases}
		u_{tt} = 4\Delta u, \\
		u|_{t=0} = 2\cos y,\quad u_t|_{t=0} = x - 3y,
	\end{cases}
	\qquad
	u = u(t,x,y),\ (x,y)\in\mathbb{R}^2,\ t>0.
	\)
\end{cond}

\begin{sol}

\end{sol}

\begin{out}

\end{out}

\newpage
\TaskTag[done]{№1.5}
\begin{cond}
	Вычислить производные всех порядков в смысле теории обобщённых функций:

	\(
	f(x) =
	\begin{cases}
		2(x-1)^2, & x \in (0,1), \\
		0,        & x \ge 1,     \\
		1,        & x \le 0.
	\end{cases}
	\)
\end{cond}

\begin{sol}

	\Definition{Пробная-функция}{
		функция $\varphi:\mathbb{R}\to\mathbb{R}$,
		бесконечно дифференцируемая и имеющая компактный носитель ($\varphi(x)=0$ вне некоторого отрезка).
	}

	\Definition{Обобщённая функция}{
		непрерывный линейный функционал $f: D(\mathbb{R})\to\mathbb{R}$,
		$\varphi\mapsto\langle f, \varphi \rangle$,
		Здесь $D(\mathbb{R})$ -- пространство пробная-функций $\varphi$.
	}

	\Definition{Дельта-функция Дирака}{
		обобщённая функция $\delta$, определяемая действием
		$\langle \delta, \varphi \rangle = \varphi(0)$
		для любой $\varphi \in D(\mathbb{R})$.
		Для сдвинутой дельта-функции $\delta_a(x)=\delta(x-a)$ имеем
		$\langle \delta_a, \varphi \rangle = \varphi(a)$.
	}

	\Definition{Производная обобщённой функции}{
		новая обобщённая функция $f'$,
		действующая на пробная-функции $\varphi \in D(\mathbb{R})$ по правилу
		$\langle f', \varphi \rangle = -\,\langle f, \varphi' \rangle$.
		Где $f$ — исходная обобщённая функция на $\mathbb{R}$.
	}

	$f\in L_{loc}^1(\mathbb{R})$ кусочно $C^2$ и имеет разрывы лишь в точках $x=a_i$.

	$\langle f', \varphi \rangle = -\,\langle f, \varphi' \rangle
		=-\int_{-\infty}^{+\infty} f(x)\varphi'(x)dx
		=-\sum_i\int_{a_i}^{a_{i+1}} f(x)\varphi'(x)dx \\
		=-\sum_i([f(x)\varphi(x)]_{a_i}^{a_{i+1}}-\int_{a_i}^{a_{i+1}} f(x)'\varphi(x)dx)
		=\sum_i\int_{a_i}^{a_{i+1}} f(x)'\varphi(x)dx -\sum_i([f(x)\varphi(x)]_{a_i}^{a_{i+1}}) \\
		=\sum_i\int_{a_i}^{a_{i+1}} f(x)'\varphi(x)dx + \sum_i(f(a_i+0)-f(a_i-0))\varphi(a_i)
	$.

	Поскольку $\langle \delta_{a_i}, \varphi \rangle = \varphi(a_i)$,
	то $\sum_i (f(a_i+0) - f(a_i-0))\varphi(a_i) \\
		=\sum_i [f]_{a_i}\,\varphi(a_i)
		= \langle \sum_i [f]_{a_i}\delta_{a_i},\,\varphi \rangle
	$.

	Так же
	$\sum_i\int_{a_i}^{a_{i+1}} f(x)'\varphi(x)dx = \langle \{f'\}, \varphi \rangle$

	$f'=\{f'\}+\sum_i([f]_{a_i}\delta_{a_i})$.

	$f''=\{f''\}+\sum_i([f']_{a_i}\delta_{a_i})+\sum_i([f]_{a_i}\delta'_{a_i})$.

	$f^{(3)}=\{f^{(3)}\}+\sum_i([f'']_{a_i}\delta_{a_i})+\sum_i([f']_{a_i}\delta'_{a_i})+\sum_i([f]_{a_i}\delta''_{a_i})$.

	$f^{(k)}=\sum_i([f'']_{a_i}\delta^{(k-3)}_{a_i}+[f']_{a_i}\delta^{(k-2)_{a_i}}+[f]_{a_i}\delta^{(k-1)}_{a_i})$.
	$k\leq4$.
	Поскольку $C^2$.

	% начало реального решения
	$
		f(x) =
		\begin{cases}
			2(x-1)^2, & x \in (0,1), \\
			0,        & x \ge 1,     \\
			1,        & x \le 0.
		\end{cases}
	$;
	$
		\{f'\}=4(x-1), x \in (0,1)
	$;
	$
		\{f''\}=4, x \in (0,1)
	$.

	Скачки:
	$[f]_0=1$, $[f]_1=0$; $[f']_0=-4$, $[f']_1=0$; $[f'']_0=4$, $[f'']_1=-4$.

	$f'=\{f'\}+\delta(x)$.

	$f''=\{f''\}-4\,\delta(x)+\delta'(x)$

	$f^{(3)}=4\delta(x)-4\delta(x-1)-4\delta'(x)+\delta''(x).$

	$f^{(k)}=4\delta^{(k-3)}(x)-4\delta^{(k-3)}(x-1)-4\delta^{(k-2)}(x)+\delta^{(k-1)}(x)$,
	$k\ge4$.

	% конец реального решения

\end{sol}

\begin{out}

	$f'=\{f'\}+\delta(x)$.

	$f''=\{f''\}-4\,\delta(x)+\delta'(x)$

	$f^{(3)}=4\delta(x)-4\delta(x-1)-4\delta'(x)+\delta''(x).$

	$f^{(k)}=4\delta^{(k-3)}(x)-4\delta^{(k-3)}(x-1)-4\delta^{(k-2)}(x)+\delta^{(k-1)}(x)$,
	$k\ge4$.
\end{out}

\newpage
\TaskTag[todo]{№1.6}
\begin{cond}
	В пространстве \(D'(\mathbb{R})\) найти общее решение уравнения
	$y'' + 2y' + 2y = -2\delta'(x)$.
\end{cond}
