\documentclass[12pt]{article}

% --- Кодировки и язык ---
\usepackage[utf8]{inputenc}
\usepackage[T2A]{fontenc}
\usepackage[russian]{babel}

% --- Настройки страницы как в Word ---
\usepackage[a4paper,top=2cm,bottom=2cm,left=3cm,right=1.5cm]{geometry}
\usepackage{setspace}
\onehalfspacing % стандартный полуторный интервал как в Word

% --- Шрифт Times New Roman ---
\usepackage{fontspec}
\setmainfont{Times New Roman}

% --- Математика и графика ---
\usepackage{amsmath,amssymb,amsthm,mathtools}
\usepackage{graphicx}
\usepackage{tikz}
\usetikzlibrary{arrows.meta,positioning}
\tikzset{lab/.style={font=\footnotesize, inner sep=1pt}}

% --- Таблицы ---
\usepackage{multirow}
\usepackage{diagbox}
\usepackage{tabularx}

% --- Цвет и боксы ---
\usepackage{xcolor}
\usepackage[most]{tcolorbox}
\usepackage{ifthen}

% --- Гиперссылки (последним) ---
\usepackage{hyperref}

% ========================= Статус-теги =========================
\definecolor{TagTodoBack}{RGB}{180,220,255}     % TODO: голубой
\definecolor{TagProgressBack}{RGB}{255,240,140} % IN PROGRESS: жёлтый
\definecolor{TagDoneBack}{RGB}{190,255,190}     % DONE: зелёный

% \TaskTag[ todo | progress | done ]{Текст}
\newcommand{\TaskTag}[2][todo]{%
  \begingroup
  \ifthenelse{\equal{#1}{todo}}{\def\TagColor{TagTodoBack}}{%
  \ifthenelse{\equal{#1}{progress}}{\def\TagColor{TagProgressBack}}{%
  \ifthenelse{\equal{#1}{done}}{\def\TagColor{TagDoneBack}}{%
    \def\TagColor{gray!30}}}}
  \tcbox[colback=\TagColor,colframe=\TagColor,
    boxrule=0pt, arc=2pt,
    left=0pt,right=0pt,top=2pt,bottom=2pt,
    fontupper=\bfseries,
    before skip=4pt, after skip=4pt
]{#2}%
  \endgroup
}

% ===================== Структура разделов ======================

\newcommand{\subsectionstar}[1]{%
  \par\phantomsection%
  \subsection*{#1}%
  \addcontentsline{toc}{subsection}{#1}%
}

\newcommand{\Ticket}[1]{%
  \subsectionstar{Билет №#1}%
}


\newenvironment{cond}{\par\noindent\textbf{Условие.}}{\par}
\newenvironment{sol}{\par\noindent\textbf{Решение.}}{\par}
\newenvironment{out}{\par\noindent\textbf{Вывод.}}{\par}

\newcommand{\Definition}[2]{%
  \noindent\textbf{Определение.} \textit{#1} --#2
}

% =========================== Документ ==========================
\title{Случайные процессы}
\date{Сентябрь 2025}

\begin{document}
\maketitle
\tableofcontents
\newpage

\section*{Варианты}

\Ticket{0}

\TaskTag[progress]{№0.3}
\begin{cond}
	Решить краевую задачу для ограниченной струны:
	\[
		\begin{cases}
			u_{tt} = 4u_{xx} + (2 - t)\sin (\frac{\pi x}{2}), & t > 0,\; x \in (0,2), \\
			u|_{x=0} = u|_{x=2},                                                      \\
			u|_{t=0} = 3 \sin (2\pi x),\quad u_t|_{t=0} = 0.
		\end{cases}
	\]
\end{cond}

\begin{sol}

\end{sol}

\begin{out}

\end{out}

\TaskTag[progress]{№0.4}
\begin{cond}
	Найти решение задачи Коши:
	\[
		\begin{cases}
			u_{tt} = 4\Delta u, \\
			u|_{t=0} = 2xyz,\quad u_t|_{t=0} = \cosh (x),
		\end{cases}
		\qquad
		u = u(t,x,y,z),\ (x,y,z)\in\mathbb{R}^3,\ t>0.
	\]
\end{cond}

\begin{sol}

\end{sol}

\begin{out}

\end{out}

\TaskTag[progress]{№0.5}
\begin{cond}
	Вычислить производные всех порядков в смысле теории обобщённых функций:
	\[
		f(x) =
		\begin{cases}
			-4,       & x \le 0,     \\
			-(x-2)^2, & x \in (0,2), \\
			0,        & x \ge 2.
		\end{cases}
	\]
\end{cond}

\TaskTag[progress]{№0.6}
\begin{cond}
	В пространстве $D'(\mathbb{R})$ найти общее решение уравнения ($\eta(x)$ - функция Хэвисайда)
	$y'' + 4y' + 3y = -\eta(x)$.
\end{cond}

\begin{sol}

\end{sol}

\begin{out}

\end{out}
\newpage
\Ticket{1}

\TaskTag[todo]{№1.1}
\begin{cond}
	Решение с помощью формулы Даламбера краевых задач для полуограниченной струны.
\end{cond}

\TaskTag[todo]{№1.2}
\begin{cond}
	Пространство обобщённых функций $D'(\Omega)$. Примеры: регулярные обобщённые функции, $\delta$-функция.
\end{cond}

\TaskTag[todo]{№1.3}
\begin{cond}
	Решить краевую задачу для ограниченной струны:

	\(
	\begin{cases}
		u_{tt} = 9u_{xx} + (2e - 1)\sin 3x,              \\
		u|_{x=0} = u|_{x=1} = 0, & t > 0,\; x \in (0,1), \\
		u|_{t=0} = -\sin 2\pi x,\quad u_t|_{t=0} = 0.
	\end{cases}
	\)
\end{cond}

\TaskTag[progress]{№1.4}
\begin{cond}
	Найти решение задачи Коши:

	\(
	\begin{cases}
		u_{tt} = 4\Delta u, \\
		u|_{t=0} = 2\cos y,\quad u_t|_{t=0} = x - 3y,
	\end{cases}
	\qquad
	u = u(t,x,y),\ (x,y)\in\mathbb{R}^2,\ t>0.
	\)
\end{cond}

\begin{sol}

\end{sol}

\begin{out}

\end{out}

\TaskTag[todo]{№1.5}
\begin{cond}
	Вычислить производные всех порядков в смысле теории обобщённых функций:

	\(
	f(x) =
	\begin{cases}
		2(x-1)^2, & x \in (0,1), \\
		0,        & x \ge 1,     \\
		1,        & x \le 0.
	\end{cases}
	\)
\end{cond}

\TaskTag[todo]{№1.6}
\begin{cond}
	В пространстве \(D'(\mathbb{R})\) найти общее решение уравнения
	$y'' + 2y' + 2y = -2\delta'(x)$.
\end{cond}

\newpage
\Ticket{2}

\newpage
\Ticket{3}

\TaskTag[todo]{№3.1}
\begin{cond}
	Метод Фурье для волнового уравнения в пространстве размерности $2$ (прямоугольник).
\end{cond}

\TaskTag[todo]{№3.2}
\begin{cond}
	Понятие обобщённой производной. Связь обобщённой производной и производной в смысле теории обобщённых функций. Пространства Соболева $H^1(\Omega)$ и $\dot{H}^1(\Omega)$, их полнота.
\end{cond}

\TaskTag[todo]{№3.3}
\begin{cond}
	Решить краевую задачу для ограниченной струны:

	\(
	\begin{cases}
		u_{tt} = 16u_{xx} + (3 - e^{5t})\cos \pi x,          \\
		u_x|_{x=0} = u_x|_{x=2} = 0, & t > 0,\; x \in (0,2), \\
		u|_{t=0} = -2\cos 3\pi x,\quad u_t|_{t=0} = 0.
	\end{cases}
	\)
\end{cond}

\TaskTag[todo]{№3.4}
\begin{cond}
	Найти решение задачи Коши:

	\(
	\begin{cases}
		u_{tt} = 16\Delta u, \\
		u|_{t=0} = e^{2x - y},\quad u_t|_{t=0} = 3y^2,
	\end{cases}
	\qquad
	u = u(t,x,y),\ (x,y)\in\mathbb{R}^2,\ t>0.
	\)
\end{cond}

\TaskTag[todo]{№3.5}
\begin{cond}
	Вычислить производные всех порядков в смысле теории обобщённых функций:

	\(
	f(x) =
	\begin{cases}
		-(x - 2)^2, & x \in (0,2), \\
		0,          & x \ge 2,     \\
		-1,         & x \le 0.
	\end{cases}
	\)
\end{cond}

\TaskTag[todo]{№3.6}
\begin{cond}
	В пространстве \(D'(\mathbb{R})\) найти общее решение уравнения
	($\eta(x)$ - функция Хэвисайда)
	$y'' + 4y' + 4y = -2\eta(x)$.
\end{cond}

\newpage
\Ticket{4}

\TaskTag[todo]{№4.1}
\begin{cond}
\end{cond}

\TaskTag[todo]{№4.2}
\begin{cond}
\end{cond}

\TaskTag[todo]{№4.3}
\begin{cond}
	Решить краевую задачу для ограниченной струны:

	\(
	\begin{cases}
		u_{tt} = u_{xx} + t e^{t}\cos\!\frac{x}{2},            \\
		u_x|_{x=0} = u|_{x=\pi} = 0, & t > 0,\; x \in (0,\pi), \\
		u|_{t=0} = 0,\quad u_t|_{t=0} = 4\cos\!\frac{3x}{2}.
	\end{cases}
	\)
\end{cond}

\TaskTag[todo]{№4.4}
\begin{cond}
	Найти решение задачи Коши:

	\(
	\begin{cases}
		u_{tt} = \Delta u + (t+1)^2, \\
		u|_{t=0} = -\cos 3x\,\sinh y,\quad u_t|_{t=0} = 0,
	\end{cases}
	\qquad
	u = u(t,x,y),\ (x,y)\in\mathbb{R}^2,\ t>0.
	\)
\end{cond}

\TaskTag[todo]{№4.5}
\begin{cond}
	Вычислить производные всех порядков в смысле теории обобщённых функций:

	\(
	f(x) =
	\begin{cases}
		4x - x^2, & x \in (0,2), \\
		4,        & x \ge 2,     \\
		1,        & x \le 0.
	\end{cases}
	\)
\end{cond}

\TaskTag[todo]{№4.6}
\begin{cond}
	В пространстве \(D'(\mathbb{R})\) найти общее решение уравнения
	$y'' - 4y' + 3y = 2\delta'(x)$.
\end{cond}

\newpage
\Ticket{5}
\newpage
\Ticket{6}

\TaskTag[todo]{№6.1}
\begin{cond}
\end{cond}

\TaskTag[todo]{№6.2}
\begin{cond}
\end{cond}

\TaskTag[todo]{№6.3}
\begin{cond}
	Решить краевую задачу для ограниченной струны:

	\(
	\begin{cases}
		u_{tt} = 4u_{xx} + \cos 2t \,\sin 3x,                                               \\
		u|_{x=0} = u|_{x=\frac{\pi}{2}} = 0, & t > 0,\; x \in \left(0,\frac{\pi}{2}\right), \\
		u|_{t=0} = 0,\quad u_t|_{t=0} = 2\sin x.
	\end{cases}
	\)
\end{cond}

\TaskTag[todo]{№6.4}
\begin{cond}
	Найти решение задачи Коши:

	\(
	\begin{cases}
		u_{tt} = 4\Delta u, \\
		u|_{t=0} = 2\cos y,\quad u_t|_{t=0} = x - 3y,
	\end{cases}
	\qquad
	u = u(t,x,y),\ (x,y)\in\mathbb{R}^2,\ t>0.
	\)
\end{cond}

\TaskTag[todo]{№6.5}
\begin{cond}
	Вычислить производные всех порядков в смысле теории обобщённых функций:

	\(
	f(x) =
	\begin{cases}
		2(x+1)^2, & x \in (0,1), \\
		0,        & x \ge 1,     \\
		3,        & x \le 0.
	\end{cases}
	\)
\end{cond}

\TaskTag[todo]{№6.6}
\begin{cond}
	В пространстве \(D'(\mathbb{R})\) найти общее решение уравнения
	$y'' - 2y' + y = 3\eta(x)$ \;(\(\eta(x)\) — функция Хэвисайда).
\end{cond}

\newpage
\Ticket{7}
\newpage
\Ticket{8}
\newpage
\Ticket{9}

\TaskTag[todo]{№9.1}
\begin{cond}
	Принцип Дюамеля. Формулы Кирхгофа и Пуассона для неоднородного волнового уравнения.
\end{cond}

\TaskTag[todo]{№9.2}
\begin{cond}
	Пространство основных функций $D(\Omega)$, примеры основных функций. Действия над основными функциями. Сходимость основных функций, примеры сходящихся и расходящихся последовательностей.
\end{cond}

\TaskTag[todo]{№9.3}
\begin{cond}
	Решить краевую задачу для ограниченной струны:

	\(
	\begin{cases}
		u_{tt} = 9u_{xx} + (2t - 1)\sin 3\pi x,          \\
		u|_{x=0} = u|_{x=1} = 0, & t > 0,\; x \in (0,1), \\
		u|_{t=0} = -\sin 2\pi x,\quad u_t|_{t=0} = 0.
	\end{cases}
	\)
\end{cond}

\TaskTag[todo]{№9.4}
\begin{cond}
	Найти решение задачи Коши:

	\(
	\begin{cases}
		u_{tt} = \Delta u + (t+1)^2, \\
		u|_{t=0} = -\cos 3x\,\sinh y,\quad u_t|_{t=0} = 0,
	\end{cases}
	\qquad
	u = u(t,x,y),\ (x,y)\in\mathbb{R}^2,\ t>0.
	\)
\end{cond}

\TaskTag[todo]{№9.5}
\begin{cond}
	Вычислить производные всех порядков в смысле теории обобщённых функций:

	\(
	f(x) =
	\begin{cases}
		4x - x^2, & x \in (0,2), \\
		3,        & x \ge 2,     \\
		0,        & x \le 0.
	\end{cases}
	\)
\end{cond}

\TaskTag[todo]{№9.6}
\begin{cond}
	В пространстве \(D'(\mathbb{R})\) найти общее решение уравнения ($\eta(x)$ - функция Хэвисайда)
	$y'' - 9y = -\eta(x)$.
\end{cond}

\newpage
\Ticket{10}

\TaskTag[todo]{№10.1}
\begin{cond}
	Распространение волн в пространствах различных размерности. Качественные различия.
\end{cond}

\TaskTag[todo]{№10.2}
\begin{cond}
	Построение фундаментального решения обыкновенного линейного дифференциального уравнения.
\end{cond}

\TaskTag[todo]{№10.3}
\begin{cond}
	Решить краевую задачу для ограниченной струны:

	\(
	\begin{cases}
		u_{tt} = 4u_{xx} + \cos 2t \,\sin 3x,                                               \\
		u|_{x=0} = u|_{x=\frac{\pi}{2}} = 0, & t > 0,\; x \in \left(0,\frac{\pi}{2}\right), \\
		u|_{t=0} = 0,\quad u_t|_{t=0} = 2\sin x.
	\end{cases}
	\)
\end{cond}

\TaskTag[todo]{№10.4}
\begin{cond}
	Найти решение задачи Коши:

	\(
	\begin{cases}
		u_{tt} = 25\Delta u - e^{-t}y, \\
		u|_{t=0} = 0,\quad u_t|_{t=0} = (x+1)^2 (y-1)^2,
	\end{cases}
	\qquad
	u = u(t,x,y),\ (x,y)\in\mathbb{R}^2,\ t>0.
	\)
\end{cond}

\TaskTag[todo]{№10.5}
\begin{cond}
	Вычислить производные всех порядков в смысле теории обобщённых функций:

	\(
	f(x) =
	\begin{cases}
		3(x+1)^2, & x \in (-1,0), \\
		3,        & x \ge 0,      \\
		1,        & x \le -1.
	\end{cases}
	\)
\end{cond}

\TaskTag[todo]{№10.6}
\begin{cond}
	В пространстве \(D'(\mathbb{R})\) найти общее решение уравнения
	$y'' + 4y = -4\delta(x)$.
\end{cond}


\end{document}
